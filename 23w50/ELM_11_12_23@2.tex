\documentclass[a4]{article}
\usepackage[utf8]{inputenc}
%\usepackage[czech]{babel}
\usepackage{amsmath}
\usepackage{amssymb}
\usepackage{graphicx}
\usepackage{listings}

\usepackage{geometry}

 \geometry{
 a4paper,
 total={170mm,257mm},
 left=25mm,
 top=25mm,
 }

\begin{document}
\title{Určení materiálu neznámého vodiče}
\author{} % Autor
\date{}

\title{Přímé a nepřímé měření odporu}

\section{Úkol měření}
\begin{enumerate}
    \item V rámci domácí přípravy nastudujte problematiku měrného elektrického odporu.
    \item Do závěrečných poznámek zpracujte v rámci domácí přípravy přehled hodnot měrného elektrického odporu běžných kovů.
    \item Změřte hodnoty jednotlivých vodičů a určete materiál, ze kterého jsou vyrobeny. Měřte
    \begin{itemize}
        \item přímou metodou (RLC můstek, multimetr);
        \item Ohmovou metodou;
        \item paralelní srovnávací metodou;
        \item sériovou srovnávací metodou;
        \item Wheatstoneovým můstkem.
    \end{itemize}
    \item Naměřené a vypočtené hodnoty zaneste do tabulky a porovnejte.
    \item Vyjádřete procentuální odchylky naměřených hodnot a srovnejte s hodnotami udanými výrobcem.
\end{enumerate}

\section{Obecná část}%

\section{Schéma zapojení}

\section{Postup měření}
\begin{enumerate}
    \item
    \item
    \item
    \item
    \item
\end{enumerate}


\section{Tabulky naměřených hodnot}
\subsection{můstek}
vodic 3 → 11,25 Ohm, 0,06 A, 7,4 V
vodič → 2,5 Ohm, 0,11 A, 12,5 V

\section{Výpočty a odvození}
Zde proveďte vzorová dosazení pro jednotlivé výpočty.

\section{Tabulky vypočtených hodnot}


\section{Poznámky}

\section{Záznam naměřených hodnot}



\section{Odpovědi na otázky}

\section{Závěr}

\section{Informační prameny použité pro zpracování protokolu}
\begin{enumerate}
    \item {\dotfill}
    \item {\dotfill}
    \item {\dotfill}
    \item {\dotfill}
    \item {\dotfill}
    \item {\dotfill}
\end{enumerate}


\section{Použité přístroje}

\section{Hodnocení}


\end{document}
